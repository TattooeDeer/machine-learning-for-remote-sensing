\secnumberlesssection{INTRODUCCIÓN}
Las imágenes adquiridas mediante teledetección son ampliamente utilizadas para aplicaciones tales como agricultura, uso de terrenos, estudio de vegetación o vigilancia militar, sin embargo, resulta difícil obtener imágenes de alta resolución, tanto espectral como espacial, debido a las limitaciones de implementar sistemas con estas capacidades. Desde el punto de vista espectral, se puede distinguir dos tipos de sensores implementados en los satélites actuales que dan nombre a imágenes con resoluciones homónimas: \textbf{multiespectrales} e \textbf{hiperespectrales}, donde las últimas poseen una mayor resolución espectral, permitiendo mayores aplicaciones que su contra parte multiespectral, pero siendo más escasas y costosas.

Un producto de teledetección (imagen), ya sea multiespectral o hiperespectral, se conforma al menos de dos dimensiones espaciales y una espectral. Al hablar de resolución espacial se hace referencia a la relación entre el área de un píxel y la correspondiente área terrestre observada, por lo que al referirse a este tipo de resolución se utilizan escalas en metros $[m]$. Por otro lado, la resolución espectral hace referencia al espectro de luz empírico medido por el sensor, se habla de \textit{bandas} en este contexto para referirse a segmentos del espectro de luz continuo que el sensor captura de forma limitada. Un sensor hiperespectral no solo posee más bandas que uno multiespectral, sino que además cada una de sus bandas suelen representar intervalos más pequeños del espectro, resultando en información más precisa sobre la radiancia/reflectancia en dichas longitudes de onda.

Dentro del contexto anterior, nace el problema de la \textbf{"superresolución"}, el cual busca sintetizar una imagen de alta resolución a partir de uno o más productos de baja resolución. En el caso de este trabajo, se trata dicho problema para el caso espectral.

El problema de la superresolución espectral entonces, implica ser capaz de estimar el comportamiento de la función de radiancia/reflectancia empíricamente observada para la escena y obtener las bandas que se debiesen observar si se hubiese utilizado un satélite hiperespectral.

--------------------------------

Debe proporcionar a un lector los antecedentes suficientes para poder contextualizar en general la situación tratada, a través de una descripción breve del área de trabajo y del tema particular abordado, siendo bueno especificar la naturaleza y alcance del problema; así como describir el tipo de propuesta de solución que se realiza, esbozar la metodología a ser empleada e introducir a la estructura del documento mismo de la memoria.

En el fondo, que el lector al leer la Introducción pueda tener una síntesis de cómo fue desarrollada la memoria, a diferencia del Resumen dónde se explicita más qué se hizo, no cómo se hizo.
